\chapter{Predicate Logic}
\section{System Specifications}
\begin{enumerate}
    \item \(\forall u. m\)
    \item \(\forall p. l \rightarrow a\)
    \item \(f \leftrightarrow p\)
    \item \(\exists r. t \land p \rightarrow r\)
\end{enumerate}

\section{Rules of Inference}
\begin{proof}
    \begin{equation*}
        \begin{array}{rll}
            \text{1.} & (\neg r \lor \neg f) \rightarrow (s \land l) & \text{Premise}                       \\
            \text{2.} & s \rightarrow t                              & \text{Premise}                       \\
            \text{3.} & \neg t                                       & \text{Premise}                       \\
            \text{4.} & \neg(\neg r \lor \neg f) \lor (s \land l)    & \text{Law of conditionals using (1)} \\
            \text{5.} & (r \land f) \lor (s \land l)                 & \text{De Morgan's on (4)}            \\
            \text{6.} & \neg t \rightarrow \neg s                    & \text{Contrapositive of (2)}         \\
            \text{7.} & \neg s                                       & \text{Modus ponens from (3) and (6)} \\
            \text{7.} & r \land f                                    & \text{Domination law on (4)}         \\
            \text{8.} & r                                            & \text{Simplification on (7)}         \\
        \end{array}
    \end{equation*}
\end{proof}


\section{Direct Proof}
\begin{proof}
    \begin{theorem}
        The sum of two even integers is even.
    \end{theorem}
    \begin{enumerate}
        \item Let the even integers \(a = 2p\), and \(b = 2q\).
        \item Assume the statement is true: \(a + b = 2r\).
        \item \(2p + 2q = 2r\).
        \item Factorise the LHS: \(2(p + q) = 2r\).
        \item Let \(r = p + q\).
        \item \(2r = 2r\).
    \end{enumerate}
\end{proof}
