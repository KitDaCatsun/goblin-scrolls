\chapter{Set Theory}

\section{Sets}
\begin{enumerate}
    \item true
    \item true
    \item false
    \item true
    \item false
    \item true
    \item true
    \item true
    \item true
    \item false
\end{enumerate}

\section{Subsets}
\begin{proof}
    Suppose that \(A\), \(B\), and \(C\) are sets such that \(A \subseteq B\) and \(B \subseteq C\). Show that \(A \subseteq C\).
    \begin{itemize}
        \item Let \(x\) be an arbitrary element of \(A\).
        \item Because \(A \subseteq B\), \(x \in B\).
        \item Because \(B \subseteq C\), \(x \in C\).
        \item As \(x\) is in \(A\) and \(x \in C\), and \(x\) is an arbitrary element, \(A \subseteq C\).
    \end{itemize}
\end{proof}

\tocsection{Power Set}
\tocsection{Set Operations}

\section{Set Identities}
\begin{proof}
    Prove the second De Morgan law by showing that if A and B are sets then:
    \begin{equation*}
        \overline{A \cup B} = \overline{A} \cap \overline{B}
    \end{equation*}
    \begin{itemize}
        \item If \(x \in \overline{A \cup B}\), \(x\) is not in \(A \cup B\).
        \item Therefore, \(x \in \overline{A}\) and \(x \in \overline{B}\).
        \item Therefore \(x \in \overline{A} \cap \overline{B}\), and so \(\overline{A \cup B} \subseteq \overline{A} \cap \overline{B}\).
        \item Suppose \(x \in \overline{A} \cap \overline{B}\).
        \item Therefore, \(x\) is not in \(A\) and \(x\) is not in \(B\).
        \item Therefore, \(x\) is not in \(A \cup B\).
        \item Therefore, \(x \in \overline{A \cup B}\).
        \item Therefore, \(\overline{A} \cap \overline{B} \subseteq \overline{A \cup B}\).
        \item Because \(\overline{A \cup B} \subseteq \overline{A} \cap \overline{B}\) and \(\overline{A} \cap \overline{B} \subseteq \overline{A \cup B}\), \(\overline{A \cup B} = \overline{A} \cap \overline{B}\).
    \end{itemize}
\end{proof}

\tocsection{Properties of Functions}

\section{Function Composition}
Suppose that \(g\) is a function from \(A\) to \(B\) and \(f\) is a function from \(B\) to \(C\). Prove
each of these statements.
\begin{proof}
    If \(f \circ g\) is onto, then \(f\) must also be onto.
    \begin{itemize}
        \item Let \(z \in C\)
        \item Because \(f \circ g\) is onto, there is an element \(x \in A\) such that \(f(g(x)) = z\)
        \item Then \(g(x)\) is an element of \(B\) that maps to \(z\) under the function \(f\) which shows that \(f\) is unto.
    \end{itemize}
\end{proof}
\begin{proof}
    If \(f \circ g\) is one-to-one, then \(g\) must also be one-to-one.
    \begin{itemize}
        \item Let \(x_1\) and \(x_2\) be elements of \(A\) then assume \(f(g(x_1)) = f(g(x_2))\).
        \item Such that \(g(x_1) = g(x_2)\).
        \item Because \(f \circ g\) is one-to-one, ths implies that \(x_1 = x_2\) and shows that \(g\) is one-to-one.
    \end{itemize}
\end{proof}
\begin{proof}
    If \(f \circ g\) is a bijection, then \(g\) is onto if and only if \(f\) is one-to-one.
    \begin{itemize}
        \item Let \(x_1\) and \(x_2\) be elements of \(A\) such that \(g(x_1) = y_1\) and \(g(x_2) = y_2\).
        \item Suppose \(g\) is onto, and let \(y_1\) and \(y_2\) be elements of \(B\) such that \(f(y_1) = f(y_2)\).
        \item Then if \((f \circ g)(x_1) = f(g(x_1)) = f(y_1) = f(y_2) = f(g(x_2)) = (f \circ g)(x_2)\) and because \(f \circ g\) is one-to-one \(x_1 = x_2\).
        \item Therefore \(y_1 = y_2\), and so \(f\) is one-to-one.
    \end{itemize}
    \begin{itemize}
        \item Suppose that \(f\) is one-to-one.
        \item To show that \(g\) is onto, let \(y \in B\).
        \item Because \(f \circ g\) is onto, there is an element \(x \in A\) such that \(f(g(x)) = f(y)\).
        \item Because \(f\) is one-to-one, this means that \(y = g(x)\), and so \(g\) is onto.
    \end{itemize}
\end{proof}
