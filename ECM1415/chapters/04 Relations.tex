\chapter{Relations}

\section{Relations}
\subsection{Relations}
Relations allow for the representation of many-to-many mappings, unlike functions.

The ordered \(n\)-tuple \((a_1, a_2, \dots, a_n)\) is the ordered collection that has \(a_1\) as its first element, \(a_2\) as its second, ect. \(2\)-tuples are called ordered pairs. Two tuples are equal if they have the same elements in the same order.

\subsubsection{Cartesian Product}
The cartesian product of the sets \(A_1, A_2, \dots, A_n\), denoted \(A_1 \times A_2 \times \dots \times A_n\), is the set of ordered \(n\)-tuples \((a_1, a_2, \dots, a_n)\) where \(a_i\) belongs to \(A_i\) for \(i = 1, \dots, n\):
\begin{equation*}
    A_1 \times A_2 \times \dots A_n = \{(a_1, a_2, \dots, a_n) | a_i \in A_i \text{ for } i = 1, 2, \dots, n\}
\end{equation*}

\subsubsection{Binary Relations}
A binary relation \(R\) from a set \(A\) to a set \(B\) is a subset \(R \subseteq A \times B\).
\begin{example}
    Let \(A = {0, 1, 2}\) and \(B = \{a, b\}\), then:
    \(\{(0, a), (0, b), (1, a), (2, b)\}\) is a relation from \(A\) to \(B\)
\end{example}

\subsection{Forming new relations}
The set operations \(\cup\), \(\cap\), and \(-\), can be used on relations.

The composition of two functions \(R_2 \circ R_1\), where \(R_1\) maps \(A\) to \(B\), and \(R_2\) from \(B\) to \(C\), is defined as:
\begin{equation*}
    \{(a, c) | \exists b \in B\ldotp (a, b) \in R_1 \land (b, c) \in R_2\}
\end{equation*}
The powers \(R^n\) of a relation \(R\) is \(R \circ R \circ \dots \circ R\) \(n\) times.

\section{Properties of relations}
\subsection{Properties}
\subsubsection{Reflexivity}
A relation is reflexive if and only if \((a, a) \in R\) for every element \(a \in A\):
\begin{equation*}
    \forall x \ldotp (x \in A \rightarrow (x, x) \in R)
\end{equation*}
\begin{examples}
    These relations on \(\mathbb{Z}\) are reflexive:
    \begin{itemize}
        \item \(\{(a, b) | a = b\}\)
        \item \(\{(a, b) | a \leq b\}\)
    \end{itemize}
\end{examples}

\subsubsection{Symmetry}
A relation \(R\) is symmetric if and only if \((b, a) \in R\) whenever \((a, b) \in R\):
\begin{equation*}
    \forall x\ldotp \forall y\ldotp ((x, y) \in R \rightarrow (y, x) \in R)
\end{equation*}

\subsubsection{Antisymmetry}
A relation \(R\) on a set \(A\) such that for all \(a, b \in A\), if \((a, b) \in R\) and \((b, a) \in R\) then \(a = b\) is called antisymmetric\footnote{Symmetry and antisymmetry are not related: one relation can be both.}:
\begin{equation*}
    \forall x, y \ldotp ((x, y) \in R \land (y, x) \in R \rightarrow x = y)
\end{equation*}
\begin{examples}
    The following relations on \(\mathbb{Z}\) are antisymmetric:
    \begin{itemize}
        \item \(\{(a, b) | a = b\}\)
        \item \(\{(a, b) | a < b\}\)
    \end{itemize}
\end{examples}

\subsubsection{Transitivity}
A relation \(R\) is called transitive if whenever \((a, b) \in R\) and \((b, c) \in R\), then \((a, c) \in R\) for all \(a, b, c \in A\):
\begin{equation*}
    \forall x, y, z \ldotp ((x, y) \in R \land (y, z) \in R \rightarrow (x, z) \in R)
\end{equation*}

\subsection{Closures}
If \(R\) is a relation on a set \(A\), then the closure of \(R\) with respect to \(P\), if it exists, is the relation \(S\) on \(A\) such that:
\begin{itemize}
    \item It contains \(R\).
    \item It satisfies property \(P\).
    \item It is a subset of every subset of \(A \times A\) containing \(R\) with property \(P\).
\end{itemize}
\begin{example}
    The reflexive closure of \(R = \{(a, b) | a < b\}\) on \(\mathbb{Z}\) is:
    \begin{equation*}
        R \cup \Delta = \{(a, b) | a < b\} \lor \{(a, a) | a \in A\} = \{(a, b) | a \leq b\}
    \end{equation*}
\end{example}

\begin{example}
    THe symmetric closure of \(R = \{(a, b) | a < b\}\) on \(\mathbb{Z}\) is:
    \begin{equation*}
        R \cup R^{-1} = \{(a, b) | a > b\} \cup \{(b, a) | a > b\} = \{(a, b) | a \neq b\}
    \end{equation*}
\end{example}

\begin{example}
    Let \(R = \{(1, 3),(1, 4),(2, 1),(3, 2)\}\). Finding the transitive closure of \(R\) requires recursively adding missing elements until the set is transitive.
\end{example}

\section{Equivalence Relations}
\subsection{Equivalence Relations}
Equivalence relations are reflexive, symmetric, and transitive. Elements \(a\) and \(b\) related by an equivalence relation are called equivalent, denoted \(a ~ b\).

\begin{example}
    If \(R\) is the relation on the set of strings of ENglish letters such that \(a R b\) if and only if \(l(a) = l(b)\), where \(l(x)\) is the length of the string \(x\).
    \begin{itemize}
        \item Reflexive because \(l(a) = l(a)\), it follows that \(a R a\) for all strings \(a\).
        \item Symmetric: suppose that \(a R b\). Since \(l(a) = l(b)\), \(l(b) = l(a)\) and thus \(b R a\).
        \item Transitivity: Suppose that \(a R b\) and \(b R c\). Since \(l(a) = l(b)\) and
              \(l(b) = l(c)\) we have \(l(a) = l(a)\) and thus \(a R c\).
    \end{itemize}
\end{example}
If \(R\) is an equivalence on a set \(A\), the set of all elements related to an element \(a\) of \(A\) is called the equivalence class of \(a\).

The equivalence class of \(a\) with respect to \(R\) is denoted by \([a]_R = \{s | (a, s) \in R\}\). If \(b \in [a]_R\) then \((b\) is called a representative of this equivalence class.
