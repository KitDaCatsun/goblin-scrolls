\chapter{Set Theory}
\section{Sets}
\subsection{Basic Notation}
A set is an unordered collection of objects. The objects in a set are called its elements or members. A set is said to contain all its elements. The notation \(a \in A\) denotes that \(a\) is in set \(A\), \(a \notin A\) denotes that it is not.

The rooster method of denoting a set is \(S = \{a, b, c, d\}\). Ellipses can be used when a pattern is clear: \(S = \{a, b, c, \dots, \}\).

\begin{example}
    Sets include:
    \begin{itemize}
        \item Vowels in the english alphabet: \(V = \{a, e, i, o, u\}\).
        \item Integers less than \(0\): \(O = \{..., -3, -2, -1\}\).
    \end{itemize}
\end{example}
Set-builder notation specifies the properties that all members must satisfy.
\begin{example}
    \begin{equation*}
        S = {x | x \text{ is a positive integer less than } 100}
    \end{equation*}
    \begin{align*}
        [a, b] & = \{x | a \leq x \leq b\} \\
        (a, b) & =\{x | a < x < b\}        \\
        [a, b) & =\{x | a \leq x < b\}
    \end{align*}
\end{example}
Some important sets are:
\begin{align*}
    \mathbb{N}      & = \{0, 1, 2, 3, \dots\}                                                       \\
    \mathbb{Z}      & = \{\dots, -1, 0, 1, \dots\}                                                  \\
    \mathbb{Z^+}    & = \{1, 2, 3, \dots\}                                                          \\
    \mathbb{Q}      & = \{\frac{q}{p} | p \in \mathbf{Z}, q \in \mathbf{Z}, \text{ and } q \ne 0 \} \\
    \mathbb{R}      & = \text{all real numbers}                                                     \\
    \mathbf{U}      & = \text{the set containing everything under consideration}                    \\
    \emptyset, \{\} & = \text{the empty set}
\end{align*}
The cardinality of a set is the number of elements in a set. If there are \(n \in \mathbf{N}\) distinct elements in \(S\), \(S\) is finite. If \(n \notin \mathbf{N}\), the set is infinite and has no cardinality. The cardinality of set \(S\) is denoted \(|S|\).

\subsection{Set equality}
Two sets \(A\) and \(B\) are equal if and only if they have the same elements:
\begin{equation}
    \label{eq:set_equivalence}
    A = B \leftrightarrow \forall x\ldotp (x \in A \leftrightarrow x \in B)
\end{equation}
The order and repetitions do not affect equivalence.

\subsection{Subsets}
The set \(A\) is a subset of \(B\), denoted \(A \subseteq B\), if and only if every element of \(A\) is also in \(B\):
\begin{equation*}
    \label{eq:subset}
    A \subseteq B \leftrightarrow \forall x\ldotp (x \in A \rightarrow x \in B)
\end{equation*}
Note how this is true even if \(A = B\). If \(A \neq B\) as well, then \(A\) is a proper subset (\(\subset\)) of \(B\).

The power set, \(\wp(A)\), of a set \(A\) is the set that contains all subsets of set \(A\). \(|\wp(A)| = 2^{|A|}\)\footnote{Except \(|\wp(\emptyset)| = 1\).}, and will always include \(\emptyset\) and \(A\).
\begin{example}
    \begin{align*}
        A      & = \{a, b\}                              \\
        \wp(A) & = \{\emptyset, \{a\}, \{b\}, \{a, b\}\}
    \end{align*}
\end{example}
To prove that \(A \subseteq B\), find an element \(x\) such that \(x \in A\) with \(x \notin B\) (proof by counterexample).

\section{Set operations}
\subsection{Basic operations}
The union, \(\cup\), of sets \(A\) and \(B\), is the set:
\begin{equation*}
    A \cup B = \{x | x \in A \lor x \in B\}
\end{equation*}
The intersection, \(\cap\), of sets \(A\) and \(B\), is the set:
\begin{equation*}
    A \cap B = \{x | x \in A \land x \in B\}
\end{equation*}
The difference, \(\backslash\), of sets \(A\) and \(B\), is the set:
\begin{equation*}
    A \backslash B = \{x | x \in A \land x \notin B\}
\end{equation*}
The complement, \(\overline{A}\), of a set \(A\), is the set:
\begin{equation*}
    U - A = A' = \overline{A} = \{x | x \in A \land x \notin A\}
\end{equation*}

\subsection{Set identities}
Identities:
\begin{align}
    A \cup \emptyset & = A \\
    A \cap U         & = A
\end{align}
Domination:
\begin{align}
    A \cup U         & = U         \\
    A \cap \emptyset & = \emptyset
\end{align}
Idempotent:
\begin{align}
    A \cup A & = A \\
    A \cap A & = A
\end{align}
Complementation:
\begin{align}
    \overline{\overline{A}} = A
\end{align}
Commutative:
\begin{align}
    A \cup B & = B \cup A \\
    A \cap B & = B \cap A
\end{align}
Associative:
\begin{align}
    A \cup (B \cup C) & = (A \cup B) \cup C \\
    A \cap (B \cap C) & = (A \cap B) \cap C
\end{align}
Distributive:
\begin{align}
    A \cap (B \cup C) & = (A \cap B) \cup (A \cap C) \\
    A \cup (B \cap C) & = (A \cup B) \cap (A \cup C)
\end{align}
De Morgan's:
\begin{align}
    \overline{A \cup B} & = \overline{A} \cap \overline{B} \\
    \overline{A \cap B} & = \overline{A} \cup \overline{B}
\end{align}
Absorption:
\begin{align}
    A \cup (A \cap B) & = A \\
    A \cap (A \cup B) & = A \\
\end{align}

\subsection{Proving Identities}
If \(A \subseteq B \land B \subseteq A\), then \(A = B\).

\begin{example}
    Show that \(\overline{A \cap B} = \overline{A} \cup \overline{B}\).
    \begin{proof}
        \(\overline{A} \cup \overline{B} \subseteq \overline{A \cap B}\)
        \begin{enumerate}
            \item Assume \(x \in \overline{A \cap B}\).
            \item Therefore \(x \notin A \cap B\).
            \item Therefore \(x \notin A \lor x \notin B\).
            \item Therefore \(x \in \overline{A} \lor x \in \overline{B}\).
            \item Therefore \(x \in \overline{A} \cup \overline{B}\).
        \end{enumerate}
    \end{proof}
    The proof continues by showing \(\overline{A \cap B} \subseteq \overline{A} \cup \overline{B}\) in a similar way.
\end{example}

Equational reasoning can also be used to derive new identities from known ones.

\subsection{Membership tables}
Membership tables are the set equivalent of a truth table. A \(1\) means an element is in the column's set, a \(0\) means it is not.

\begin{example}
    Membership table of \(A \cup B\):

    \begin{tabular}{ccc}
        \toprule
        \(A\) & \(B\) & \(A \cup B\) \\
        \midrule
        0     & 0     & 0            \\
        0     & 1     & 1            \\
        1     & 0     & 1            \\
        1     & 1     & 1            \\
        \bottomrule
    \end{tabular}
\end{example}

\section{Functions}
\section{Basic Notation}
A function \(f\) from a nonempty set \(A\) to a nonempty set \(B\) is an assignment of each element of \(A\) to exactly one element of \(B\). A function is denoted:
\begin{equation*}
    f: A \rightarrow B
\end{equation*}
\(f(a) = b\) means \(b\) is the unique element of \(B\) assigned by the function \(f\) to the element \(a\) of \(A\).

In this case, \(A\) is the domain and \(B\) is the codomain. \(b\) is the image of \(a\) under \(f\), and \(a\) is the preimage. The range of \(f\), denoted \(f(A)\), is the set of all images of points in \(A\) under \(f\).

Two functions are equal if they have the same domain, the same codomain, and map each element of the domain to the same element of the codomain.

A function is said to be one-to-one or injective if and only if \(f(a) = f(b)\) implies that \(a = b\) for all \(a\) and \(b\) in the domain of \(f\).

A function \(f\) is called surjective if and only if for every element \(b \in B\) there is an element \(a \in A\) with \(f(a) = b\).

A function is a one-to-one correspondence or bijective, if it is both surjective and injective. That is, every \(b \in B\) is the image of exactly one \(a \in A\), for every \(b\).

To prove the properties of a function \(f: A \rightarrow B\):
\begin{itemize}
    \item To show that \(f\) is injective assume \(f(x) = f(y)\) and show \(x = y\).
    \item To show that \(f\) is not injective find \(x, y \in A\) such that \(x \neq y\) and \(f(x) = f(y)\).
    \item To show that \(f\) is surjective consider an arbitrary element \(y \in B\) and find an element \(x \in A\) such that \(f(x) = y\).
    \item To show that \(f\) is not surjective find \(y \in B\) such that \(f(x) \neq y\) for all \(x \in A\).
\end{itemize}
