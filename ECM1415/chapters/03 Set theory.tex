\chapter{Set Theory}
\section{Sets}
\subsection{Basic Notation}
A set is an unordered collection of objects. The objects in a set are called its elements or members. A set is said to contain all its elements. The notation \(a \in A\) denotes that \(a\) is in set \(A\), \(a \notin A\) denotes that it is not.

The rooster method of denoting a set is \(S = \{a, b, c, d\}\). Ellipses can be used when a pattern is clear: \(S = \{a, b, c, \dots, \}\).

\begin{example}
    Sets include:
    \begin{itemize}
        \item Vowels in the english alphabet: \(V = \{a, e, i, o, u\}\).
        \item Integers less than \(0\): \(O = \{..., -3, -2, -1\}\).
    \end{itemize}
\end{example}
Set-builder notation specifies the properties that all members must satisfy.
\begin{example}
    \begin{equation*}
        S = {x | x \text{ is a positive integer less than } 100}
    \end{equation*}
    \begin{align*}
        [a, b] & = \{x | a \leq x \leq b\} \\
        (a, b) & =\{x | a < x < b\}        \\
        [a, b) & =\{x | a \leq x < b\}
    \end{align*}
\end{example}
Some important sets are:
\begin{align*}
    \mathbf{N}      & = \{0, 1, 2, 3, \dots\}                                                       \\
    \mathbf{Z}      & = \{\dots, -1, 0, 1, \dots\}                                                  \\
    \mathbf{Z^+}    & = \{1, 2, 3, \dots\}                                                          \\
    \mathbf{Q}      & = \{\frac{q}{p} | p \in \mathbf{Z}, q \in \mathbf{Z}, \text{ and } q \ne 0 \} \\
    \mathbf{R}      & = \text{all real numbers}                                                     \\
    \mathbf{U}      & = \text{the set containing everything under consideration}                    \\
    \emptyset, \{\} & = \text{the empty set}
\end{align*}
The cardinality of a set is the number of elements in a set. If there are \(n \in \mathbf{N}\) distinct elements in \(S\), \(S\) is finite. If \(n \notin \mathbf{N}\), the set is infinite and has no cardinality. The cardinality of set \(S\) is denoted \(|S|\).

\subsection{Set equality}
Two sets \(A\) and \(B\) are equal if and only if they have the same elements:
\begin{equation}
    \label{eq:set_equivalence}
    A = B \leftrightarrow \forall x. (x \in A \leftrightarrow x \in B)
\end{equation}
The order and repetitions do not affect equivalence.

\subsection{Subsets}
The set \(A\) is a subset of \(B\), denoted \(A \subseteq B\), if and only if every element of \(A\) is also in \(B\):
\begin{equation*}
    \label{eq:subset}
    A \subseteq B \leftrightarrow \forall x. (x \in A \rightarrow x \in B)
\end{equation*}
Note how this is true even if \(A = B\). If \(A \neq B\) as well, then \(A\) is a proper subset (\(\subset\)) of \(B\).

The power set, \(\wp(A)\), of a set \(A\) is the set that contains all subsets of set \(A\). \(|\wp(A)| = 2^{|A|}\)\footnote{Except \(|\wp(\emptyset)| = 1\).}, and will always include \(\emptyset\) and \(A\).
\begin{example}
    \begin{align*}
        A      & = \{a, b\}                              \\
        \wp(A) & = \{\emptyset, \{a\}, \{b\}, \{a, b\}\}
    \end{align*}
\end{example}
To prove that \(A \subseteq B\), find an element \(x\) such that \(x \in A\) with \(x \notin B\) (proof by counterexample).

\section{Set operations}
\subsection{Basic operations}
The union, \(\cup\), of sets \(A\) and \(B\), is the set:
\begin{equation*}
    A \cup B = \{x | x \in A \lor x \in B\}
\end{equation*}
The intersection, \(\cap\), of sets \(A\) and \(B\), is the set:
\begin{equation*}
    A \cap B = \{x | x \in A \land x \in B\}
\end{equation*}
The difference, \(\backslash\), of sets \(A\) and \(B\), is the set:
\begin{equation*}
    A \backslash B = \{x | x \in A \land x \notin B\}
\end{equation*}
The complement, \(\overline{A}\), of a set \(A\), is the set:
\begin{equation*}
    U - A = A' = \overline{A} = \{x | x \in A \land x \notin A\}
\end{equation*}
