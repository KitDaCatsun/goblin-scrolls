\chapter{Logic and Proof}

\section{Propositional Logic}

\subsection{Propositions}
A proposition is a declarative sentence that is either true or false, for example:
\begin{itemize}
    \item The moon is made of green cheese.
    \item \(1 + 0 = 0\)
    \item \(0 + 0 = 2\)
\end{itemize}
Statements that are not proposition include:
\begin{itemize}
    \item ``Sit down"
    \item \(x + 1 = 2\)
\end{itemize}

\subsection{Propositional Logic}
Atomic propositions use variables (\(p\), \(q\), \(r\), \dots) and constants (\(\mathbf{T}\) and \(\mathbf{F}\)). Compound propositions use logic operators.

\subsubsection{Negation}
The negation (`not') of a proposition \(p\) is given by \(\neg p\). \(\neg p\) is true if \(p\) is false.

\begin{table}[htbp]
    \centering
    \begin{tabular}{cc}
        \toprule
        \(p\)          & \(\neg p\)     \\
        \midrule
        \(\mathbf{T}\) & \(\mathbf{F}\) \\
        \(\mathbf{F}\) & \(\mathbf{T}\) \\
        \bottomrule
    \end{tabular}
    \caption{Negation Truth Table}
\end{table}

\subsubsection{Conjunction}
The conjunction (`and') of propositions \(p\) and \(q\) is denoted by \(p \land q\). \(p \land q\) is true only if both \(p\) and \(q\) are true.

\begin{table}[htbp]
    \centering
    \begin{tabular}{ccc}
        \toprule
        \(p\)          & \(q\)          & \(p \land p\)  \\
        \midrule
        \(\mathbf{F}\) & \(\mathbf{F}\) & \(\mathbf{F}\) \\
        \(\mathbf{F}\) & \(\mathbf{T}\) & \(\mathbf{F}\) \\
        \(\mathbf{T}\) & \(\mathbf{F}\) & \(\mathbf{F}\) \\
        \(\mathbf{T}\) & \(\mathbf{T}\) & \(\mathbf{T}\) \\
        \bottomrule
    \end{tabular}
    \caption{Conjunction Truth Table}
\end{table}

\subsubsection{Disjunction}
The conjunction (`or') of propositions \(p\) and \(q\) is denoted by \(p \lor q\). \(p \lor q\) is true as long as at least one of \(p\) or \(q\) are true.

\begin{table}[htbp]
    \centering
    \begin{tabular}{ccc}
        \toprule
        \(p\)          & \(q\)          & \(p \lor p\)   \\
        \midrule
        \(\mathbf{F}\) & \(\mathbf{F}\) & \(\mathbf{F}\) \\
        \(\mathbf{F}\) & \(\mathbf{T}\) & \(\mathbf{T}\) \\
        \(\mathbf{T}\) & \(\mathbf{F}\) & \(\mathbf{T}\) \\
        \(\mathbf{T}\) & \(\mathbf{T}\) & \(\mathbf{T}\) \\
        \bottomrule
    \end{tabular}
    \caption{Disjunction Truth Table}
\end{table}

\subsubsection{Exclusive Or}
The exclusive or of propositions \(p\) and \(q\) is denoted by \(p \oplus q\). \(p \oplus q\) is true if \(p\) is true or \(q\) is true, but not both.

\begin{table}[htbp]
    \centering
    \begin{tabular}{ccc}
        \toprule
        \(p\)          & \(q\)          & \(p \oplus p\) \\
        \midrule
        \(\mathbf{F}\) & \(\mathbf{F}\) & \(\mathbf{F}\) \\
        \(\mathbf{F}\) & \(\mathbf{T}\) & \(\mathbf{T}\) \\
        \(\mathbf{T}\) & \(\mathbf{F}\) & \(\mathbf{T}\) \\
        \(\mathbf{T}\) & \(\mathbf{T}\) & \(\mathbf{F}\) \\
        \bottomrule
    \end{tabular}
    \caption{Exclusive Or Truth Table}
\end{table}
