\chapter{Logic and Proof}

\section{Propositional Logic}

\subsection{Propositions}
A proposition is a declarative sentence that is either true or false, for example:
\begin{itemize}
    \item The moon is made of green cheese.
    \item \(1 + 0 = 0\)
    \item \(0 + 0 = 2\)
\end{itemize}
Statements that are not proposition include:
\begin{itemize}
    \item ``Sit down"
    \item \(x + 1 = 2\)
\end{itemize}

\subsection{Propositional Logic}
Atomic propositions use variables (\(p\), \(q\), \(r\), \dots) and constants (\(\mathbf{T}\) and \(\mathbf{F}\)). Compound propositions use logic operators.

\subsubsection{Negation}
The negation (`not') of a proposition \(p\) is given by \(\neg p\). \(\neg p\) is true if \(p\) is false.

\begin{table}[htbp]
    \centering
    \begin{tabular}{cc}
        \toprule
        \(p\)          & \(\neg p\)     \\
        \midrule
        \(\mathbf{T}\) & \(\mathbf{F}\) \\
        \(\mathbf{F}\) & \(\mathbf{T}\) \\
        \bottomrule
    \end{tabular}
    \caption{Negation Truth Table}
\end{table}

\subsubsection{Conjunction}
The conjunction (`and') of propositions \(p\) and \(q\) is denoted by \(p \land q\). \(p \land q\) is true only if both \(p\) and \(q\) are true.

\begin{table}[htbp]
    \centering
    \begin{tabular}{ccc}
        \toprule
        \(p\)          & \(q\)          & \(p \land q\)  \\
        \midrule
        \(\mathbf{F}\) & \(\mathbf{F}\) & \(\mathbf{F}\) \\
        \(\mathbf{F}\) & \(\mathbf{T}\) & \(\mathbf{F}\) \\
        \(\mathbf{T}\) & \(\mathbf{F}\) & \(\mathbf{F}\) \\
        \(\mathbf{T}\) & \(\mathbf{T}\) & \(\mathbf{T}\) \\
        \bottomrule
    \end{tabular}
    \caption{Conjunction Truth Table}
\end{table}

\subsubsection{Disjunction}
The conjunction (`or') of propositions \(p\) and \(q\) is denoted by \(p \lor q\). \(p \lor q\) is true as long as at least one of \(p\) or \(q\) are true.

\begin{table}[htbp]
    \centering
    \begin{tabular}{ccc}
        \toprule
        \(p\)          & \(q\)          & \(p \lor q\)   \\
        \midrule
        \(\mathbf{F}\) & \(\mathbf{F}\) & \(\mathbf{F}\) \\
        \(\mathbf{F}\) & \(\mathbf{T}\) & \(\mathbf{T}\) \\
        \(\mathbf{T}\) & \(\mathbf{F}\) & \(\mathbf{T}\) \\
        \(\mathbf{T}\) & \(\mathbf{T}\) & \(\mathbf{T}\) \\
        \bottomrule
    \end{tabular}
    \caption{Disjunction Truth Table}
\end{table}

\subsubsection{Exclusive Or}
The exclusive or of propositions \(p\) and \(q\) is denoted by \(p \oplus q\). \(p \oplus q\) is true if \(p\) is true or \(q\) is true, but not both.

\begin{table}[htbp]
    \centering
    \begin{tabular}{ccc}
        \toprule
        \(p\)          & \(q\)          & \(p \oplus q\) \\
        \midrule
        \(\mathbf{F}\) & \(\mathbf{F}\) & \(\mathbf{F}\) \\
        \(\mathbf{F}\) & \(\mathbf{T}\) & \(\mathbf{T}\) \\
        \(\mathbf{T}\) & \(\mathbf{F}\) & \(\mathbf{T}\) \\
        \(\mathbf{T}\) & \(\mathbf{T}\) & \(\mathbf{F}\) \\
        \bottomrule
    \end{tabular}
    \caption{Exclusive Or Truth Table}
\end{table}

\subsection{Compound propositions}
If \(p\) and \(q\) are propositions, then \(p \rightarrow q\) is a conditional statement or implication. \(p\) is the hypothesis (or antecedent or premise), and \(q\) is the conclusion (or consequence).

\begin{table}[htbp]
    \centering
    \begin{tabular}{ccc}
        \toprule
        \(p\)          & \(q\)          & \(p \rightarrow q\) \\
        \midrule
        \(\mathbf{F}\) & \(\mathbf{F}\) & \(\mathbf{T}\)      \\
        \(\mathbf{F}\) & \(\mathbf{T}\) & \(\mathbf{T}\)      \\
        \(\mathbf{T}\) & \(\mathbf{F}\) & \(\mathbf{F}\)      \\
        \(\mathbf{T}\) & \(\mathbf{T}\) & \(\mathbf{T}\)      \\
        \bottomrule
    \end{tabular}
    \caption{Implication Truth Table}
\end{table}

Note how if the hypothesis is false, the implication always holds.

If \(p \rightarrow q\), `If it is raining I stay at home':
\begin{itemize}
    \item \(q \rightarrow p\) is the converse: `If I stay at home it is raining'.
    \item \(\neg q \rightarrow \neg p\) is the contrapositive: `If it is not raining I will not stay at home'.
    \item \(\neg p \rightarrow \neg q\) is the inverse: 'If I do not stay at home it is not raining.'.
\end{itemize}
If \(p\) and \(q\) are propositions, \(p \leftrightarrow  q\) is the biconditional `if and only if'.

\begin{table}[htbp]
    \centering
    \begin{tabular}{ccc}
        \toprule
        \(p\)          & \(q\)          & \(p \leftrightarrow q\) \\
        \midrule
        \(\mathbf{F}\) & \(\mathbf{F}\) & \(\mathbf{T}\)          \\
        \(\mathbf{F}\) & \(\mathbf{T}\) & \(\mathbf{F}\)          \\
        \(\mathbf{T}\) & \(\mathbf{F}\) & \(\mathbf{F}\)          \\
        \(\mathbf{T}\) & \(\mathbf{T}\) & \(\mathbf{T}\)          \\
        \bottomrule
    \end{tabular}
    \caption{Biconditional Truth Table}
\end{table}

A truth table can be constructed for compound propositions by creating a row for every combination of values for the atomic propositions, and a column for the truth value of every sub-expression.

A proposition that is always true, like \(p \lor \neg p\), is called a tautology. A contradiction like \(p \land \neg p\) is always false. A statement that can be either is a contingency.

The logical operators have the precedence:
\begin{enumerate}
    \item Negation \(\neg\)
    \item Conjunction \(\land\)
    \item Disjunction \(\lor\)
    \item Implication \(\rightarrow\)
    \item Biconditional \(\leftrightarrow\)
\end{enumerate}

\section{Propositional equivalences}

\subsection{Equivalence}
Two propositions \(p\) and \(q\) are equivalent if they have the same truth tables, shown by \(p \equiv q\). This is the same as saying that \(p \leftrightarrow q\) is a tautology.

The contrapositive of an implication is equivalent to the implication, but the inverse and converse are not.

\subsection{Key logical equivalences}
% TODO
The identity law says that true is the identity of conjunction, and false the identity of disjunction:
\begin{align}
    p \land \mathbf{T} & \equiv p \\
    p \lor \mathbf{F}  & \equiv p
\end{align}
The domination law says that true dominates disjunction, and false dominates conjunction:
\begin{align}
    p \land \mathbf{F} & \equiv \mathbf{F} \\
    p \lor \mathbf{T}  & \equiv \mathbf{T}
\end{align}
The idempotent law says that the conjunction or disjunction of \(p\) with itself is \(p\):
\begin{align}
    p \land p & \equiv p \\
    p \lor p  & \equiv p
\end{align}
The double negative law says that two negatives do not change a variable:
\begin{align}
    \neg(\neg p) & \equiv p
\end{align}
The negation law says that the disjunction of \(p\) with \(\neg p\) is true, and the conjunction is false:
\begin{align}
    p \land \neg p & \equiv \mathbf{F} \\
    p \lor \neg p  & \equiv \mathbf{T}
\end{align}
The commutative law says that disjunction and conjunction are commutative:
\begin{align}
    p \land q & \equiv q \land p \\
    p \lor q  & \equiv q \lor p
\end{align}
The associative law says that the order that three conjunctions or disjunctions are evaluated does not matter:
\begin{align}
    p \land (q \land r) & \equiv (p \land q) \land r \\
    p \lor (q \lor r)   & \equiv (p \lor q) \lor r
\end{align}
The distributive law says that a disjunction or conjunction can be distributed across the other operator:
\begin{align}
    p \land (q \lor r) & \equiv p \land q \lor p \land r    \\
    p \lor (q \land r) & \equiv (p \lor q) \land (p \lor r)
\end{align}
The absorption law says that:
\begin{align}
    p \lor (p \land q) & \equiv p \\
    p \land (p \lor q) & \equiv p
\end{align}
De Morgan's laws state:
\begin{align}
    \neg(p \land q) & \equiv \neg p \lor \neg q  \\
    \neg(p \lor q)  & \equiv \neg p \land \neg q
\end{align}
Laws for conditional statements are:
\begin{align}
    p \rightarrow q & \equiv \neg p \lor q             \\
    p \rightarrow q & \equiv \neg q \rightarrow \neg p
\end{align}
And for biconditional statements:
\begin{align}
    p \leftrightarrow q & \equiv (p \rightarrow q) \lor (q \rightarrow p)
\end{align}

\subsection{Constructing new logical equivalences}
To prove that \(A \equiv B\), produce a series of equivalences that begins with \(A\) and ends with \(B\).
