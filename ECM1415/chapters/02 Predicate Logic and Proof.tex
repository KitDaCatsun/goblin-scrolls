\chapter{Predicate Logic and Proof}

\section{Predicates and Quantifiers}
\subsection{Predicate Logic}
A predicate is a proposition that contains variables. Variables denote subjects with \(x\), \(y\), and \(z\). Predicates denote a property of a subject: \(P(x)\), \(M(x)\). Quantifiers quantify over subjects, for example \(\forall x\) and \(\exists x\).

\begin{example}
    Let \(P(x)\) denote \(x > 0\), then:
    \begin{itemize}
        \item \(P(3) \lor P(-1)\) is true
        \item \(P(3) \land P(-1)\) is false
        \item \(P(3 \land P(y))\) is not a proposition
    \end{itemize}
\end{example}
Predicates with multiple variables are denoted \(Q(x, y)\).

\subsection{Quantifiers}
The universal quantifier \(\forall\) means `for all'. \(\forall x\ldotp P(x)\) asserts that \(P(x)\) is true for every \(x\) in the domain. The existential quantifier \(\exists\) means `there exists'. \(\exists x\ldotp P(x)\) asserts that \(P(x)\) is true for some \(x\) in the domain\footnote{\(\exists!\) is sometimes used to denote `there exists exactly one'.}. Quantifiers have higher precedence than all logical quantifiers.

The domain is often denoted by \(U\), for universe of discourse.

\begin{example}
    If \(P(x)\) denotes `\(x\) is even', and \(U\) is the integers, then:
    \begin{itemize}
        \item \(\forall x\ldotp P(x)\) is false
        \item \(\exists x\ldotp P(x)\) is true
    \end{itemize}
\end{example}
The truth values of a predicate with a domain depends on both the domain and the predicate.

Quantifiers can be nested, so that a predicate with multiple quantifiers can be written. The statement `for every \(x\), there is a \(y\)\dots' is denoted \(\forall x \exists y\ldotp \dots\). The order of the quantifiers is important. For a proposition with the nested quantifiers \(\exists y \forall x\) to be true, the predicate must hold for some values of \(y\), no matter the value of \(x\).

\subsection{Equivalence}
Two statements involving predicates and quantifiers are logically equivalent if and only if they have the same truth value for every predicate substituted into these statements and for every domain used for the variables in the expressions.

De Morgan's laws for quantifiers are:
\begin{align*}
    \neg(\forall x\ldotp P(x)) & \equiv \exists x\ldotp \neg P(x) \\
    \neg(\exists x\ldotp P(x)) & \equiv \forall x\ldotp \neg P(x)
\end{align*}

\section{Rules of inference}
\subsection{Arguments}
An argument is a sequence of propositions.
\begin{itemize}
    \item All but the final proposition are called premises.
    \item The last statement is the conclusion.
    \item The argument is valid if the premises imply the conclusion.
    \item An argument form is an argument that is valid no matter what propositions are substituted into its propositional variables.
\end{itemize}

\begin{example}
    An argument:
    \begin{equation*}
        \begin{array}{r l}
                       & Snowing \rightarrow Study \\
                       & Snowing                   \\
            \cline{2-2}
            \therefore & Study
        \end{array}
    \end{equation*}
    In argument form:
    \begin{equation*}
        \begin{array}{r l}
                       & p \rightarrow q \\
                       & p               \\
            \cline{2-2}
            \therefore & q
        \end{array}
    \end{equation*}
    The corresponding tautology is:
    \begin{equation*}
        (p \land (p \rightarrow q)) \rightarrow q
    \end{equation*}
\end{example}

\subsection{Common Rules}
Modus ponens:
\begin{equation}
    \begin{array}{r l}
                   & p \rightarrow q \\
                   & p               \\
        \cline{2-2}
        \therefore & q
    \end{array}
\end{equation}
Modus tollens:
\begin{equation}
    \begin{array}{r l}
                   & p \rightarrow q \\
                   & \neg p          \\
        \cline{2-2}
        \therefore & \neg q
    \end{array}
\end{equation}
Hypothetical syllogism:
\begin{equation}
    \begin{array}{r l}
                   & p \rightarrow q \\
                   & q \rightarrow r \\
        \cline{2-2}
        \therefore & p \rightarrow r
    \end{array}
\end{equation}
Disjunctive syllogism:
\begin{equation}
    \begin{array}{r l}
                   & p \lor q \\
                   & \neg p   \\
        \cline{2-2}
        \therefore & q
    \end{array}
\end{equation}
Addition:
\begin{equation}
    \begin{array}{r l}
                   & p        \\
        \cline{2-2}
        \therefore & p \lor q
    \end{array}
\end{equation}
Simplification:
\begin{equation}
    \begin{array}{r l}
                   & p \land q \\
        \cline{2-2}
        \therefore & p
    \end{array}
\end{equation}
Conjunction:
\begin{equation}
    \begin{array}{r l}
                   & p         \\
                   & q         \\
        \cline{2-2}
        \therefore & p \land q
    \end{array}
\end{equation}
Resolution:
\begin{equation}
    \begin{array}{r ll}
                   & \neg  p \lor r \\
                   & p \lor q       \\
        \cline{2-2}
        \therefore & q \lor r
    \end{array}
\end{equation}

\subsection{Rules of inference}
A valid argument is a sequence of statements such that each statement is either a premise or follows from the previous statements by rules of inference.

\begin{example}
    Show that from \(p \land (p \rightarrow q)\) follows \(q\):

    \begin{equation*}
        \begin{array}{rll}
            \text{1.} & p \land (p \rightarrow q) & \text{Premise}                        \\
            \text{2.} & p                         & \text{Simplification using (1)}       \\
            \text{3.} & p \rightarrow q           & \text{Simplification using (1)}       \\
            \text{4.} & q                         & \text{Modus Ponens using (2) and (3)}
        \end{array}
    \end{equation*}

\end{example}

\subsection{Quantifiers}
Universal instantiation:
\begin{equation}
    \begin{array}{r l}
                   & \forall x\ldotp P(x) \\
        \cline{2-2}
        \therefore & P(c)
    \end{array}
\end{equation}
Universal generalization:
\begin{equation}
    \begin{array}{r l}
                   & P(c) \text{ for an arbitrary } c \\
        \cline{2-2}
        \therefore & \forall x\ldotp P(x)
    \end{array}
\end{equation}

Existential instantiation:
\begin{equation}
    \begin{array}{r l}
                   & \exists x\ldotp P(x)             \\
        \cline{2-2}
        \therefore & P(c) \text{ for some element } c
    \end{array}
\end{equation}
Existential generalization:
\begin{equation}
    \begin{array}{r l}
                   & P(c) \text{ for some element } c \\
        \cline{2-2}
        \therefore & \exists x\ldotp P(x)
    \end{array}
\end{equation}

\section{Proofs}
\subsection{Proofs}
A proof is a valid argument that establishes the truth of a statement. In informal proofs:
\begin{itemize}
    \item More than one rule of inference is used in a step.
    \item Steps may be skipped.
    \item The rules of inference are not explicitly stated.
\end{itemize}
A theorem is an important statement that can be proven. A lemma is a result that is needed to prove a theorem. A corollary is a result which follows directly from a theorem. A conjecture is a statement that is being proposed to be true.

An proof is constructed by selecting a proof strategy and then using axioms, definitions, theorems, lemmas, and rules of inference.

\subsection{Proof strategies}
Strategies for conditional statements \(p \rightarrow q\):
\begin{itemize}
    \item Trivial proof if \(q\) is true.
    \item Vacuous proof if \(p\) is false.
    \item Direct proof: assume \(p\), then show \(q\).
    \item Proof by contraposition: assume \(\neg p\), then show \(\neg q\).
    \item Proof by cases: Find \(p_1, \dots, p_n\) such that \(p \equiv p_1 \lor \dots \lor p_n\), and show \(p_1 \rightarrow q, \dots, p_n \rightarrow q\).
\end{itemize}
For other statements:
\begin{itemize}
    \item Proof by contradiction: to prove \(p\), assume \(\neg p\) and show false.
    \item To prove \(p \leftrightarrow q\), assume \(p\) and show \(q\) and assume \(q\) to show \(p\).
    \item To prove \(\forall x\ldotp P(x)\) is false, find an example \(x\) for which \(P(x)\) is false.
\end{itemize}
Existence proofs:
\begin{itemize}
    \item Constructive: Find a witness \(a\) such that \(P(a)\).
\end{itemize}
